\section{Problem Formulation}
% <400-600 words>
% 615 words so far
\subsection{Game Description}
\subsubsection{Rules of Yahtzee}
\textit{Yahtzee} is played with five standard six-sided dice and a shared scorecard containing 13 categories.
Turns are rotated among players. A turn starts with a player rolling all five dice. They may then choose to keep
some dice, re-rolling the remaining ones. This process can be repeated two more times, for a total of three total rolls.
After the final roll, the player must select one of the 13 scoring categories to apply to their current dice.
Each category has specific scoring rules, and each can only be used once per game.

\subsubsection{Mathematical Representation of Yahtzee}
\label{sec:yahtzee-definitions}
The space of all possible dice configurations is:
$$\mathcal{D} \in \{1, 2, 3, 4, 5, 6\}^5$$
and the current state of the dice is represented as:
\begin{equation}
    \mathbf{d} \in \mathcal{D}
\end{equation}

In addition, we can represent the score card as a vector of length 13, where each element corresponds to a scoring category:
\begin{equation}
    \mathbf{c} = (c_1, c_2, \ldots, c_{13}) \text{ where } c_i \in \mathcal{D}_i \cup \{\varnothing\}
\end{equation}
where $\varnothing$ indicates an unused category.

Let us also define a dice face counting function which we can use to simplify score calculations:
\begin{align}
    n_v(\mathbf{d})        & = \sum_{i=1}^{5} \mathbb{I}(d_i = v),
    \quad v \in \{1,\dots,6\}                                                 \nonumber \\
    \mathbf{n}(\mathbf{d}) & = \big(n_1(\mathbf{d}),\dots,n_6(\mathbf{d})\big)
\end{align}

Let the potential score for each category be defined as follows (where detailed scoring rules can be found in Appendix~\ref{app:scoring}):
\begin{equation}
    \begin{aligned}
        \mathbf{f}(\mathbf{d}) & =
        \bigl(f_1(\mathbf{d}), f_2(\mathbf{d}), \ldots, f_{13}(\mathbf{d})\bigr)
    \end{aligned}
\end{equation}

The current turn number can be represented as:
\begin{equation}
    t \in \{1, 2, \ldots, 13\}, \quad t = \sum_{i=1}^{13} \mathbb{I}(c_i \neq \varnothing)
\end{equation}

A single turn is composed of an initial dice roll, two optional re-rolls, and a final scoring decision.
Let $r = 0$, with $r \in \{0,1,2\}$ which is the number of rolls taken so far.

Prior to the first roll, the dice are randomized:

$$
    \mathbf{d}_{r=0} \sim U(\mathcal{D})
$$

The player must decide which dice to keep and which to re-roll. Let the player define a keep vector:
\begin{equation}
    \mathbf{k} \in \{0,1\}^5
\end{equation}
where $\mathbf{k}_i = 1$ indicates that die $i$ is kept, otherwise it is re-rolled.

We can then define the transition of the dice state after a re-roll as:
\begin{align*}
    \mathbf{d}' & \sim U(\mathcal{D}),                          \\
    \mathbf{d}_{r+1}
                & = (\mathbf{1} - \mathbf{k}) \odot \mathbf{d}'
    + \mathbf{k} \odot \mathbf{d}
\end{align*}


When $r=2$, the player must choose a scoring category to apply their current dice to. Define a scoring choice mask as a one-hot vector:
\begin{equation}
    \mathbf{s} \in \{0,1\}^{13}, \quad \|\mathbf{s}\|_1 = 1
\end{equation}

For the purposes of calculating the final (or current) score, any field that has not been scored yet can be counted as zero.
We can define a mask vector for this:

\begin{align}
     & \mathbf{u}(\mathbf{c}) \in \{0,1\}^{13}                                                                  \nonumber \\
     & \mathbf{u}(\mathbf{c})_i = \mathbb{I}\bigl(c_i \neq \varnothing\bigr), \quad \forall i = \{ 1, \ldots 13 \}
\end{align}


If a player achieves a total score of 63 or more in the upper section (categories 1-6), they receive a bonus of 35 points:
$$
    B(\mathbf{c}) = \begin{cases}
        35, & \sum_{i=1}^{6} \mathbf{u}(\mathbf{c})_i \cdot \mathbf{c}_i \geq 63 \\
        0,  & \text{otherwise}
    \end{cases}
$$

There is an additional "Joker" bonus, detailed in Appendix~\ref{app:scoring}.

The player's score can thus be calculated as:
\begin{equation}
    \mathrm{score}(\mathbf{c}) = B(\mathbf{c}) + \big\langle \mathbf{u}(\mathbf{c}), \mathbf{c} \big\rangle
\end{equation}

\subsection{MDP Formulation}
We model \textit{Yahtzee} as a Markov Decision Process $(\mathcal S,\mathcal A,P,R,\gamma)$ \citep{Puterman1994MDP}.

A state is represented as $\mathbf{s} = (\mathbf{d},\mathbf{c},r, t)$, where $\mathbf{d}$ is the current
dice configuration, $\mathbf{c}$ the scorecard, and $r$ the roll index, and $t$ the current turn index
(see Section~\ref{sec:yahtzee-definitions}).

The action is $\mathbf{a} = (\mathbf{k}, \mathbf{s})$, where $\mathbf{k}$ is the keep vector and $s$ is the score category choice.
This can be restated as a parameterization of the policy: $\pi_{\theta}(\mathbf{a}|\mathbf{s}) = \pi_{\theta}(\phi(\mathbf{s}))$,
where $\phi(\mathbf{s})$ is a feature representation of the state $\mathbf{s}$.

The transition function $P$ is is specified in Appendix~\ref{app:transition-function}.

The reward is the change in total score between steps $R_t = \mathrm{score}(c_{t+1}) - \mathrm{score}(c_t)$.

Since we desire to maximize total score at the end of the game, $\gamma = 1$.

\subsection{Single-Turn Optimization Task}
In the single-turn optimization task, the agent is trained to maximize the expected score over a single turn.
This task has 3 steps total; after being initialized with a random dice roll, the agent chooses which
dice to keep and which to re-roll twice, and then selects a scoring category. A single reward is given at the end of the turn.

This is a useful subproblem to study, as it isolates the decision-making process in a single turn,
this allows us to quickly iterate on architecture and training choices with shorter training times and without the complications of long-term credit assignment.

\subsection{Full-Game Optimization Task}
In the full-game optimization task, 13-turn episodes (totalling 39 individual steps) are played to completion.
The objective again is to maximize the total score at the end of the game.
This task is more challenging due to the longer horizon and increased variance.
Additionally, the network must learn to balance optimal single-turn play with long-term strategies, such as planning for the upper bonus.
