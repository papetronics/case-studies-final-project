%
% CSML Final Project Paper
%

\documentclass[11pt,a4paper]{article}
\usepackage[hyperref]{acl2019}
\usepackage{times}
\usepackage{latexsym}
\usepackage{float}
\usepackage{graphicx}
\usepackage{caption}
\usepackage{pgfplots}
\pgfplotsset{compat=newest}
\usepackage{url}

\aclfinalcopy % Uncomment this line for the final submission

\newcommand\BibTeX{B\textsc{ib}\TeX}

\title{Your Paper Title Here}

\author{Your Name \\
  Your Student ID \\
  \texttt{your.email@example.com} \\}

\date{}

\begin{document}
\maketitle
\begin{abstract}
  Write your abstract here. This should be a concise summary of your work, 
  including the problem you're addressing, your approach, and key results.
  Keep it to about 150-250 words.
\end{abstract}

\section{Introduction and Research Background}

Introduce your topic here. Provide background and motivation for your work.
Cite relevant papers using \cite{example-citation}.

\subsection{Background}
Provide relevant background information and related work.
r
\subsection{Problem Statement}
Clearly state the problem you are addressing.

\subsection{Research Objectives}
Outline your research objectives and what you aim to achieve.

\section{Research and Methods}

Describe your approach and methods here.

\subsection{Methodology Overview}
Provide an overview of your research methodology and approach.

\subsection{Model Architecture}
Describe your model or approach in detail.

\subsection{Experimental Design}
Describe your experimental design and evaluation metrics.

\section{Materials and Data Sources}

Describe the dataset and materials you are using.

\subsection{Dataset Description}
Provide detailed information about your dataset, including size, characteristics, and source.

\subsection{Data Collection and Preprocessing}
Explain how data was collected and any preprocessing steps performed.

\subsection{Tools and Technologies}
List the tools, frameworks, and technologies used in your research.

\section{Results}

Present your experimental results here.

\begin{table}[H]
\begin{center}
\begin{tabular}{|l|c|c|}
\hline \textbf{Method} & \textbf{Metric 1} & \textbf{Metric 2} \\ \hline
Baseline & XX.XX & XX.XX \\
Your Method & XX.XX & XX.XX \\
\hline
\end{tabular}
\end{center}
\caption{\label{results-table} Results comparison}
\end{table}


\section{Discussion and Conclusion}

\subsection{Discussion of Results}
Discuss your findings, their implications, and how they relate to existing literature.

\subsection{Limitations}
Acknowledge the limitations of your study and approach.

\subsection{Future Work}
Discuss potential directions for future research.

\subsection{Conclusion}
Summarize your work and its contributions to the field.

\section{AI Usage Disclosure}

This paper utilized artificial intelligence tools in the following ways:

\begin{itemize}
    \item \textbf{GitHub Copilot} was used for initial typesetting assistance with the LaTeX document structure and formatting.
    \item \textbf{ChatGPT GPT-5} was used for brainstorming ideas for reinforcement learning applications in games, helping to narrow down to a particular game that was both non-trivial and interesting for this research.
\end{itemize}

All other content, including research methodology, analysis, results interpretation, and conclusions, represents original work by the author. The AI tools were used only for initial ideation and technical formatting support, not for generating substantive content or analysis.

GitHub Copilot was also used to write this section.

\section{Acknowledgments}

Acknowledge any help or resources you used.

\bibliography{references}
\bibliographystyle{acl_natbib}

\end{document}
